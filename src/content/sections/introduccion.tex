Una red neuronal artificial (RNA) es concebida como un
procesador paralelo y distribuido que consiste de un gran
n\'umero de unidades de c\'omputo, conocidas como neuronas.
La interconexi\'on masiva entre neuronas tiene como objetivo
almacenar informaci\'on que representa conocimiento
o experiencia. Esta informaci\'on se define a trav\'es de la fuerza de
interconexi\'on neuronal, o peso sin\'aptico. Las redes neuronales
tienen la capacidad de aproximar funciones. Con respecto
a la ecuaci\'on de Schr\"odinger una RNA se usa para describir
la funci\'on de onda de un sistema cu\'antico. El entrenamiento
de la RNA se lleva a cabo utilizando un proceso de aprendizaje
de tal manera que las entradas-salidas de la red satisfacen
la ecuaci\'on de Schr\"odinger. Se considera que las redes neuronales
artificiales tienen la capacidad de describir de forma
eficiente a una funci\'on propia, pero dada la gran cantidad de
m\'inimos existentes en el espacio de b\'usqueda se hace dif\'icil
encontrar la soluci\'on exacta. La respuesta de una red entrenada
depende considerablemente de los pesos iniciales independientemente
del proceso de aprendizaje utilizado.

Un algoritmo gen\'etico (AG) es una t\'ecnica de b\'usqueda
basada en estrategias evolutivas, se ha usado este tipo de
algoritmo para resolver problemas de optimizaci\'on en campos
como la qu\'imica, optimizaci\'on geom\'etrica de sistemas
rob\'oticos y mecatr\'onicos. La desventaja de un AG es su lenta
convergencia hacia la soluci\'on exacta. 

La idea es combinar una RNA y un AG con el fin de resolver
la ecuaci\'on de Schr\"odinger. La habilidad de las redes neuronales
para describir las entradas-salidas relacionadas con
conductas complejas puede ser utilizada para representar a las
funciones de onda excitadas de un sistema cu\'antico, mientras
que la ventaja de un AG con respecto a b\'usquedas globales
ayudar\'a a la RNA evitando una prematura convergencia durante
el proceso de aprendizaje.